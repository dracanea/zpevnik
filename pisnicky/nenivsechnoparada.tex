\hyt{nenivsechnoparada}
\song{Není všechno paráda} \interpret{zich}{Karel Zich}

\vers{a}{
\chord{A}Není všechno paráda, co \chord{E}hrajem,\\
\chord{D}někdy stačí \chord{E}jenom zabrn\chord{A}kat.\\
Kdo se pár not naučí a zná je,\\
k brnkání si najde třeba drát. 
}

\vers{b}{
\chord{D}Bez parády začínal jsem tenkrát,\\
\chord{E}s rozladěnou starou kytarou.\\
A trnem \chord{A}v oku byl mi školník Josef \chord{E}tenkrát,\\
když jsem \chord{D}přes plot tajně \chord{E}chodil za Klá\chord{A}rou.
}

\vers{a}{
Není všechno paráda, co cítím,\\
někdy stačí škrtnout rozumem.\\
Mám doma dvě stě dvacet silných voltů v síti,\\
proč si je mám tahat s sebou ven?
}

\vers{b}{
Bez parády svítil jsem si sirkou,\\
na zvonkový štítek u dveří.\\
Svou dívku přistihl jsem s kamarádem Jirkou,\\
od těch dob mi Jirka nevěří.
}

\vers{b}{
Není všechno paráda, co dělám,\\
i když zrovna nezpůsobím zkrat.\\
Žádná moje píseň nemusí být celá,\\
\rep{stačí půlka, aspoň chvíli mám co hrát.}
}

\newpage
