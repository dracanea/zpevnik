\hyt{marie}
\song{Marie}
\note{capo 5 (originál v F dur)}
\vspace{10mm}

\textbf{C E F G}

\vers{1}{
Je den, tak pojď, Marie, ven, budeme žít házet šutry do oken.\\
Je dva necháme doma trucovat, když nechtějí, nemusí, nebudem se vnucovat.\\
Jémine. Všechno zlý jednou pomine, tak Marie, co ti je?
}
\textbf{C E F G}

\vers{2}{
Všemocné jsou loutkařovy prsty. Ať jsou tenký nebo tlustý, občas přetrhají nit.\\
A to pak jít a nemít nad sebou svý jistý, pořád s tváří optimisty listy v žití obracet.
}

\vers{3}{
Je to jed, mazat si kolem huby med a neslyšet, jak se ti bortí svět.\\
Marie, kdo přežívá, nežije, tak ádié!
}

\vers{4}{
Marie! Už zase máš tulení sklony, jako loni, slyším kostelní zvony znít.\\
A to mě zabije, a to mě zabije, a to mě zabije. Jistojistě!
}

\refrain{
Já mám Marii rád, když má moje bytí spád.\\
Býti věčně na cestách a k ránu spícím plicím život vdechovat.\\
Nechtěj mě milovat, nechtěj mě milovat, nechtěj mě milovat.\\
Já mám Marii rád, když má moje bytí spád.\\
Býti věčně na cestách a k ránu spícím plicím život vdechovat.\\
} \textbf{C E F G}

\vers{5}{
Copak nemůže být mezi ženou a mužem přátelství, kde není nikdo nic dlužen.\\
Prostě jen prosté spříznění duší, aniž by kdokoli cokoli tušil.
}

\vers{6}{
Na na na na na na\dots
}
\newpage
