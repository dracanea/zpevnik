\hyt{minulost}
\song{Minulost} \interpret{nohavica}{Jaromír Nohavica}

\vers{1}{
\chord{D\9}Jako když k půlnoci na dveře zaklepe \chord{A}nezvaný host,\\
\chord{Em}stejně tak na tebe za rohem čeká \chord{G}tvoje minulost. \chord{A}\\
\chord{D\9}Šaty má tytéž, vlasy má tytéž a \chord{A}boty kožené,\\
\chord{Em}pomalu, kulhavě, za tebou pajdá, \chord{G}až tě dožene \chord{A}a řekne
}

\refrainn{1}{
\chord{D\9}Tak mě tu máš, \chord{A}tak si mě zvaž,\\
\chord{Em}pozvi mě dál, \chord{A}jestli mě znáš, jsem tvoje \chord{D\9}minulost.
}

\vers{2}{
V kapse máš kapesník, který jsi před lpty zavázal na uzel,\\
dávno jsi zapomněl to, co jsi koupit měl v samoobsluze.\\
Všechno, cos po cestě poztrácel, zmizelo, jak vlaky na trati,\\
ta holka nese to v batohu na zádech, ale nevrátí, jen říká
} \refsm{1}

\vers{3}{
Stromy jsou vyšší a tráva je nižší a na louce roste pýr,\\
červené tramvaje jedoucí do Kunčic svítí jak pionýr.\\
Jenom ta tvá pyšná hlava ti nakonec zůstala na šíji,\\
průvodčí, kteří tě vozili před léty dávno už nežijí.
} \refsm{}

\vers{4}{
Na větvích jabloní nerostou fíky a z kopřiv nevzroste les,\\
to, co jsi nesnědl včera a předvčírem musíš dojíst dnes.\\
Solené mandle i nasládlé hrozny máčené do medu,\\
ty sedíš u stolu a ona nese ti jídlo k obědu a říká
} \refsm{}

\vers{5}{
Rozestel postel a ke zdi si lehni, ona se přitulí,\\
dneska tě navštíví ti, kteří byli a ti, kteří už minuli.\\
Každého po jménu oslovíš, neboť si vzpomeneš na jména,\\
ráno se probudíš a ona u tebe v klubíčku schoulená, řekne ti
}

\refrainn{2}{
Tak mě tu máš, tak si mě zvaž,\\
pozvals' mě dál, teď už mě znáš, jsem tvoje minulost.
}
\newpage
