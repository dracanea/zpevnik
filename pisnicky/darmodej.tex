\hyt{darmodej}
\song{Darmoděj} \interpret{nohavica}{Jaromír Nohavica}
\ns

\vers{1}{
\chord{Am}Šel včera městem \chord{Em}muž a šel po hlavní \chord{Am}třídě. \chord{Em}\\
\chord{Am}Šel včera městem \chord{Em}muž a já ho z okna \chord{Am}viděl. \chord{Em}\\
\chord{C}Na flétnu chorál \chord{G}hrál, znělo to jako \chord{Am}zvon\\
a byl v tom všechen \chord{Em}žal, ten krásný dlouhý \chord{F}tón\\
a já jsem náhle \chord{F\kk\dm}věděl: ano, to je \chord{E\7}on, to je \chord{Am}on.
}

\vers{2}{
Vyběh' jsem do ulic jen v noční košili,\\
v odpadcích z popelnic krysy se honily\\
a v teplých postelích lásky i nelásky tiše se vrtěly rodinné obrázky\\
a já chtěl odpověď na svoje otázky, otázky.
}
\vspace{-6mm}

\refrainn{1}{
\rep{\chord{Am}Na, na na \chord{Em}na\dots\chord{C}\nc\chord{G}\nc\chord{Am}\nc\chord{F}\chord{F\kk\dm}\nc\mm\chord{E\7}}
}

\vers{3}{
Dohnal jsem toho muže a chytl za kabát,\\
měl kabát z hadí kůže, šel z něho divný chlad.\\
A on se otočil a oči plné vran a jizvy u očí, celý byl pobodán\\
a já jsem náhle věděl, kdo je onen pán, onen pán.
}

\vers{4}{
Celý se strachem chvěl, když jsem tak k němu došel\\
a v ústech flétnu měl od Hieronyma Bosche.\\
Stál měsíc nad domy, jak čírka ve vodě, jak moje svědomí, když zvrací v záchodě,\\
a já jsem náhle věděl: to je Darmoděj, můj Darmoděj.
}

\refrainn{2}{
Můj Darmoděj, vagabund osudů a lásek,\\
jenž prochází všemi sny, ale dnům vyhýbá se,\\
můj Darmoděj, krásné zlo, jed má pod jazykem,\\
když prodává po domech jehly se slovníkem.
}

\vers{5}{
Šel včera městem muž, podomní obchodník.\\
Šel, ale nejde už, krev skápla na chodník.\\
Já jeho flétnu vzal a zněla jako zvon a byl v tom všechen žal, ten krásný dlouhý tón\\
a já jsem náhle věděl: ano, já jsem on, já jsem on.
}
\vspace{-4mm}

\refrainn{3}{
Váš\dots
}
\newpage
