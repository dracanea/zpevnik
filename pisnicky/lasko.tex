\hyt{lasko}
\song{Lásko!} \interpret{kryl}{Karel Kryl}

\vers{1}{
\chord{Am}Pár zbytků pro krysy na misce od guláše,\\
\chord{E}milostný dopisy s parti\chord{Dm}í mariáše.\\
\chord{Dm}Před cestou dalekou zpocený boty zujem,\\
\chord{C}a potom po dekou \chord{E}sníme, když onanujem.
}

\refrain{
\chord{Am}Lásko, \chord{G}zavři se do pokoje,\\
\chord{Am}lásko, \chord{G}válka je holka moje,\\
s \chord{C}ní se \chord{G}milu\chord{Am}ji, když \chord{G}noci si \chord{Am}krátím. \chord{E}\\
\chord{Am}Lásko, \chord{G}slunce máš na vějíři,\\
\chord{Am}lásko, \chord{G}dvě třešně na talíři,\\
\chord{C}ty ti \chord{G}daru\chord{Am}ji, až \chord{G}jednou se \chord{Am}vrátím. \chord{E}
}

\vers{2}{
Dvacet let necelých, odznáček na baretu,\\
s úsměvem dospělých vytáhnem cigaretu.\\
V opasku u boku nabitou parabelu\footnote{Pistole Luger P.08, ve své době jedna z nejspolehlivějších samonabíjecích zbraní. Pojmenování \uv{parabela} je odvozeno od názvu náboje 9$\times$19 mm Parabellum. Jedná se o nejrozšířenější pistolový náboj na světě. Název Parabellum je odvozen z latinského \uv{Si vis pacem, para bellum.} (\uv{Hledáš-li mír, připrav se na válku.}), což bylo motto Deutsche Waffen und Munitionsfabriken, výrobce tohoto náboje.},\\
zpíváme do kroku pár metrů od bordelu.
} \refsm{}

\vers{3}{
Pár zbytků pro krysy a taška na patrony,\\
latrína s nápisy, jež nejsou pro matróny.\\
Není čas na spaní, smrtka nám drtí palce,\\
nežli se zchlastaní svalíme na kavalce.
} \refsm{}
\vspace{15pt}

\rec{\chord{E}Levá, dva.}

\refsme{}
\newpage
