\hyt{basniktenmapravoveta}
\song{Básník, ten má právo veta}

\note{capo 3}

\vers{1}{
\chord{Am}Básník, \chord{E}ten má \chord{Am}právo \chord{E}veta, \chord{Am}může s klidem \chord{E}v minu\chord{Am}tě\\
\chord{Dm}na mí\chord{A\7}stě, kde \chord{Dm}stála \chord{A\7}teta, \chord{Dm}nechat plavat \chord{A\7}labu\chord{Dm}tě.\\
\chord{G}A tu tetu \chord{C}jednou větou \chord{G}na Venuši \chord{E}zanese\\
a \chord{Am}Venu\chord{E}šané s \chord{Am}mojí \chord{E}tetou \chord{Am}budou samá \chord{E}rece\chord{Am}se, hej,\\
\chord{Am}Venu\chord{E}šané s \chord{Am}mojí \chord{E}tetou \chord{Am}budou samá \chord{E}depre\chord{Am}se.
}

\vers{2}{
Neboť tetička jim řekne fóry, co by jinde neřekla,\\
upeče jim buchet hory, což by jinde nepekla.\\
A v každé buchtě bude skryta jedna malá naděje,\\
a ten, kdo jí do polosyta, od srdce se zasměje,\\
ten, kdo jí do polosyta, od srdce se zasměje.
}

\vers{3}{
A kdo žere jako zvíře, jak ti hoši z Venuše,\\
nevejde se do talíře, a po zemi pokluše.\\
Na Venuši vzniknou zmatky -- všichni plní naděje,\\
proto vrátím tetu zpátky, ať je radši tam, kde je,\\
proto vrátím tetu zpátky, ať nám taky prospěje.
}

\vers{4}{
Básník, ten má právo veta, co se týče vesmíru,\\
ale v tomhle konci světa může psát jen na míru,\\
proto teta bude jenom tetou a Venuše Venuší\\
a blbost bude nad planetou vnášet smutek do duší,\\
a blbost bude nad planetou vnášet radost do du\chord{A}ší, juchů!
}
\newpage
