\hyt{pepita}
\song{Pepita} \interpret{plihal}{Karel Plíhal}

\vers{1}{
\chord{D}Šel jsem si koupit \chord{G}krava\chord{D}tu \chord{A}a vzal jsem \chord{Hm}první \chord{E}zprava, \chord{A}tu,\\
\chord{A}co byla křivě \chord{D}seši\chord{D\sm}tá, \chord{D}je ale \chord{Hm\7}príma, je \chord{Em}pe -\chord{A} pi\chord{D}ta.\footnote{Pepito je tkanina se střídavě světlými a tmavými barevnými efekty ve tvaru drobných kostek, jejichž velikost je dána střídou. Příbuzné vzorování mají tkaniny s tzv. kohoutí stopou a vichy.\\
Střída je část plošné textilie, která se jako vazba nebo vzor opakuje. Například střída vazby pleteniny je dána počtem řádků a sloupků. Střída vzoru je odlišná od střídy vazby.}
}

\vers{2}{
Šel jsem si koupit košili, ti by mě vyhastrošili!\\
Tak jsem jim řek', že je mi ta, zrovna ta se vzorem pepita.
}

\vers{3}{
Šel jsem si koupit na kabát, člověk se zkrátka nemá bát,\\
že něco zrovna nelítá, tak jsem si koupil pepita.
}

\vers{4}{
Šel jsem si koupit kalhoty a koupil jsem dost draho ty,\\
co se jich špína nechytá, jsou ale príma, jsou pepita.
}

\vers{5}{
Šel jsem si koupit fuchsie, a už si říkám, kup si je!\\
Tak jsem si koupil jelita, neboť řezník měl blúzu pepita.
}

\vers{6}{
Šel jsem si koupit aparát a uviděl mě kamarád.\\
Povídá k mne tichounce, \uv{Co jsi za pepitomce?}
}
\newpage
