\hyt{pohoda}
\song{Pohoda} \interpret{lenk}{Jaroslav Samson Lenk}
\vers{1}{
U nás \chord{C}v ulici bydlel pan \chord{Em}Svoboda, že se \chord{Am}pořád smál, měl přezdívku \chord{C}Pohoda.\\
Když jsem \chord{Dm}se ho ptal, jakpak \chord{G\7}dneska je, říkával: \chord{C}pohoda. \chord{Dm\7}\\
\chord{G\7}Měl \chord{C}malej byt, ženu a \chord{Em}dvě děti, v pět vstával \chord{Am}do práce\\
\vinv
domů šel \chord{C}v půl třetí, noviny \chord{Dm\7}pod paží, cukroví \chord{G\7}pro děti, zkrátka \chord{C}pohoda.\\
\chord{Dm\7}Po domě se \chord{G\7}povídalo: \chord{C}ten má asi \chord{Am}příjem,\\
\chord{Dm\7}zatímco my \chord{G\7}nádáváme, \chord{C}vesele si \chord{Am}žije\\
\chord{Dm\7}a hlavně paní \chord{G\7}Šulcová, \chord{C}ze sousedního \chord{Am}bytu,\\
\chord{D\7}poslouchala za dveřmi \chord{F}a záviděla \chord{G\7}v skrytu.
}

\vers{2}{
Paní Šulcová psala dopisy, co jsou zvláštní tím, že nemají podpisy\\
a ty úřade, a ty počti si, kdo, kde a kolik.\\
Dík paní Šulcové, jejímu dopisu, rychle pan Svoboda dostal se do spisu,\\
\vinv
ač není podpisu, je třeba prošetřit, autor se bojí.\\
Na Svobodu začali se ptát neznámí páni,\\
jak u nás ve vchodu, tak u něj v zaměstnání,\\
i když se nic nenašlo a Svoboda je čistej,\\
lepší na něj dávat pozor, kdo si má bejt jistej?
}

\vers{3}{
Včera se stěhoval pryč od nás Svoboda na nějakou samotu někde u Náchoda\\
a já vím určitě, není to náhoda, trochu se bojím.\\
Že brzo budeme i my hledat samoty k bydlení nastálo, nejen o soboty,\\
\vinv
kvůli pár Šulcovejm, no a vím na tuty, o to nestojím.\\
Usměvavej Svoboda mi totiž hrozně chybí,\\
zjišťuju to ponenáhlu a málo se mi líbí,\\
že Šulcová je na koni, a proto není náhoda,\\
\chord{Dm\7}že lidi zdravím úsměvem \chord{G\7}a odpovídám \chord{C}poho\chord{Em}da, \chord{Am}\nc\chord{G\7}\mm\sm\chord{C}poho\chord{Em}da, \chord{Am}\nc\chord{G\7}\mm\sm\chord{C}pohoda.
}
\newpage
