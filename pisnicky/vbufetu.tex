\hyt{vbufetu}
\song{V bufetu} \interpret{plihal}{Karel Plíhal}

\vers{1}{
\chord{C}Jen nechte trpaslíka \chord{Em}Ondřeje,\\
\chord{Am}jen ať si dá, co hrdlo \chord{Em}ráčí.\\
\chord{Am}Ať se, \chord{F}chudák malej, \chord{C}taky \chord{Em\7}jednou \chord{A\7}poměje,\\
\chord{D\7}všechno platím, a tak \chord{G}o co vlastně kráčí?
}

\vers{2}{
Jídlo je nejlepší lék na lásku,\\
v tomhletom mám už trochu školu.\\
Když jsem ho našel, jeho život visel na vlásku,\\
vrhnul se v zoufalosti ze psacího stolu.
}

\refrain{
\chord{F}Srdce má \chord{G}rozervané \chord{C}na ha\chord{Am}dry, \chord{Dm}pro jednu \chord{E}trpaslici \chord{Am}ze sádry,\\
\chord{A\bb\7}byla tak krásná, až se \chord{C}věřit nechtě\chord{A\7}lo, \chord{D\7}jako by ji uplácal sám \chord{G}Michelangelo.\\
\chord{F}Nosil jí \chord{G}rybízové \chord{C}korál\chord{Am}ky, \sm \chord{Dm}koukala \chord{E}nepřítomně \chord{Am}do dálky,\\
\chord{A\bb\7}tak velkou láskou ji, \chord{C}chudák, zahr\chord{A\7}nul, \chord{D\7}že by se i kámen \chord{G}nad ním ustrnul, ale sádra \chord{C}ne.
}

\vers{3}{
Tak, prosím, nechte toho Ondřeje,\\
jen ať si dá, co hrdlo ráčí.\\
Když už mu osud v lásce zrovna dvakrát nepřeje,\\
všechno platím, a tak o co vlastně kráčí?
}

\vers{4}{
Ať jsme velký nebo maličký,\\
na jedný lodi všichni plujem.\\
Koukejte, jak mu jedou laskonky a chlebíčky!\\
Než končit s životem, je lepší dostat průjem.
} \refsm{}
\newpage
