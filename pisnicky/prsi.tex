\hyt{prsi}
\song{Prší}

\refrain{
\chord{C}Prší, \chord{C\bas{H}}\nc\chord{Dm}a hvězdy \chord{G}na plakátech \chord{C}blednou, \chord{C\bas{H}}\\
\chord{Dm}zpívám si \chord{E}spolu s repro\chord{Am}bednou, \chord{Am\bas{G}}\\
jak ta \chord{F}láska deštěm \chord{G}voní,\\
stejně \chord{C}voněla i \chord{G}loni, zkrátka
}

\vers{1}{
\chord{C}prší \chord{C\bas{H}}\nc\chord{Dm}a soused \chord{G}chodí sadem s \chord{C}konví, \chord{C\bas{H}}\\
\chord{Dm}každej se \chord{E}diví, jenom \chord{Am}on ví, \chord{Am\bas{G}}\\
proč \chord{F}místo toho \chord{E}kropení si \chord{A}nezaleze k \chord{D}topení\\
a \chord{G}nepřečte si \chord{C}McBaina\footnote{Evan Hunter, známý též jako Evan McBain, rodným jménem Salvatore Lombino (1926--2005), americký spisovatel\\
a~scénárista, napsal např. scénář k filmu Alfreda Hitchcocka \textit{Ptáci}}, proč \chord{F}vozí mouku \chord{E}do mlejna.
} \refsm{}

\vers{2}{
prší a soused venku prádlo věší,\\
práce ho, jak je vidět, těší,\\
ač promáčen je na nitku, tak na co volat sanitku,\\
stejně na čísle blázince je věčně někdo na lince.
}

\cod{
\chord{C}Prší\chord{C\bas{H}}\dots\mm\chord{Dm}\nc\chord{G}\nc \textit{(fade out)}
}
\newpage
