\hyt{trhvescarborough}
\song{Trh ve Scarborough}
\hyl{scarboroughfair}{Scarborough Fair}
\ns

\vers{1}{
\chord{Am}Příteli, máš do \chord{G}Scarborough \chord{Am}jít,\\
dob\chord{C}ře \chord{Am}vím, že \chord{C}půj\chord{D}deš tam \chord{Am}rád.\\
Tam \chord{Am}dívku \chord{C}najdi na \chord{G\bas{H}}Mar\chord{Am}ket \chord{G}Street,\\
\chord{Am}co chtěla \chord{G}dřív mou ženou se \chord{Am}stát.
}

\vers{2}{
Vzkaž jí, ať šátek začne mi šít,\\
za jehlu rýč však smí jenom brát\\
a místo příze měsíční svit,\\
bude-li chtít mo ženou se stát.
}

\vers{3}{
Až přijde máj a zavoní zem,\\
šátek v písku přikaž jí prát\\
a ždímat v kvítku jabloňovém,\\
bude-li chtít mou ženou se stát.
}

\vers{4}{
Z vrkočů svých ať uplete člun,\\
v něm se může na cestu dát.\\
S tím šátkem pak ať vejde v můj dům,\\
bude-li chtít mou ženou se stát.
}

\vers{5}{
Kde útes ční nad přívaly vln,\\
zorej dva sáhy pro růží sad.\\
Za pluh ať slouží šípkový trn,\\
budeš-li chtít mým mužem se stát.
}

\vers{6}{
Osej ten sad a slzou jej skrop,\\
choď těm růžím na loutnu hrát.\\
Až začnou kvést, tak srpu se chop,\\
budeš-li chtít mým mužem se stát.
}

\vers{7}{
Z trní si lůžko zhotovit dej,\\
druhé z růží pro mě nech stát.\\
Jen pýchy své a Boha se ptej,\\
proč nechci víc tvou ženou se stát.
}
\newpage
