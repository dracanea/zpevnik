\hyt{bajecnazenska}
\song{Báječná ženská}

\vers{1}{
Tenhle \chord{E}příběh je pravda, ať visím, jestli vám budu \chord{A}lhát,\\
já jsem \chord{H\7}potkal jednu dívku a do dnešního dne ji mám \chord{E}rád.\\
Nikdy \chord{E}neměla zlost, když jsem hluboko do kapsy \chord{A}měl,\\
vždycky \chord{H\7}měla pochopení a já se s ní nikdy hádat nemu\chord{E}sel.
}

\refrain{
Když si \chord{E}báječnou ženskou vezme báječnej \chord{A}chlap,\\
tak mají \chord{H\7}báječněj život plnej báječnejch dní bez ú\chord{E}trap.\\
Celej \chord{E}den jen tak sedí a popíjejí Châteauneuf-du-\chord{A}Pape,\\
když si \chord{H\7}báječnou ženskou vezme báječnej \chord{E}chlap.
}

\vers{2}{
Nikdy jsem neslyšel \uv{kam jdeš, kdy přijdeš a kde jsi byl,}\\
a já nikdy nezapomněl, abych svoje sliby vyplnil.\\
A když vzpomínala, tak jen na to hezký, co nám život dal,\\
nedala mi příležitost, na co bych si taky stěžoval?
} \refsm{}

\vers{3}{
Tenhle příběh je pravda a sním svůj klobouk, jestli jsem vám lhal,\\
že jsem potkal jednu dívku, a tu dívku jsem si za ženu vzal.\\
Zní to jak pohádka Z oříšku královny Mab*,\\
že si báječnou ženskou vzal jeden báječnej chlap.
} \refsm{} \note{o tón výše}


\vspace*{\fill}
\note{*\footnotesize Z oříšku královny Mab - povídkové zpracování 12 her W. Shakespeara, autorkou je Eva Vrchlická}
\vspace{10pt}
\newpage
