\hyt{mnisijsoutisi}
\song{Mniši jsou tiší} \interpret{uhlirsverak}{Uhlíř \& Svěrák}
\sub{z pohádky Lotrando a Zubejda}

\vers{1}{
\chord{C}Mniši jsou \chord{Am}tiší, jsou \chord{C}tiší jak \chord{Am}myši\\
a \chord{Dm}do kroniky \chord{G}píší, že \chord{F}léta Páně \chord{G}pět, zas \chord{C}nezmě\chord{F}nil se \chord{C}svět.
}

\vers{2}{
Lidé jsou hříšní, jak v Praze, tak v Míšni\\
a letos je dost višní a svatá Lucie zas noci upije.
}

\refrain{
V \chord{F}klášteře nemají \chord{C}pohodlí, \chord{F}celý čas prakticky \chord{C}promodlí,\\
\chord{F}když utichne bzukot \chord{Em}včel, \chord{Dm}jdou spáti do prostých \chord{G}cel.
}

\vers{3}{
Mniši jsou tiší a od lidí se liší,\\
že zájmy mají vyšší a ne ty pozemské, nemyslí na ženské.
}

\vers{4}{
Dosti se postí a jen když přijdou hosti,\\
tak ve vší počestnosti si přihnou ze sklínky, však z takhle malinký.
} \refsm{}

\vers{5}{
= \mm\textbf{4.}
}

\cod{
Však z takhle malinký, však z takhle malinký.
}
\newpage
