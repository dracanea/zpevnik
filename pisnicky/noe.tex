\hyt{noe}
\song{Noe}
\nv refrain: \textbf{D A D A} \nc sloka: \textbf{G D A D\7}
\ns

\refrain{
Už v parku reznou z kyselých dešťů průlezky\\
a plíce těžknou od zplodin.\\
Svět čeká na potopu, proto je nehezky,\\
i Noe je furt bez lodi.
}
\ns

\vers{1}{
Když přišel na úřady žádat o svolení,\\
by mohl stavbu vykonat,\\
řekli mu, že už pro něj místo není,\\
kde by mohla jeho archa stát.
}
\ns

\vers{2}{
Poslední parcelu prý prodali Tescu\\
za doživotní členství v klubu.\\
Znáš svoji cenu, má generace stesku,\\
ve stopách držet krok a nepouštět hubu
}
\ns

\vers{3}{
na špacír, aby snad nemluvila pravdu.\\
Jsem taky takový, že tam, kam jde dav, jdu,\\
do sebe zavřený, spoutaný kořeny,\\
jako že žiju.
}
\ns

\vers{4}{
Předstírám štěstí, od rána budím dojem,\\
že netrpím a neprožívám nepokoje,\\
do sebe zavřený dělám,\\
jako že žiju.
}\refsm
\ns

\vers{5}{
Noe však svoje úsilí nevzdal,\\
vydal se jednou na horu,\\
k práci na lodi mu svítila hvězda\\
a v těle zahříval ho rum.
}
\ns

\vers{6}{
Postavil archu za pár hodin,\\
sešel z hor lidi do ní zvát:\\
pojďte se zachránit do mé lodi,\\
není čas ztrácet čas, ne, nemusíte se
}
\ns

\vers{7}{
bát. Najednou kolem něj bylo klubko uni\\forem, sirény vzduchem zněj, prej zklidni se ma\\
gore, už ho maj, odvezou ho a nechaj\\
pod práškama tiše spát.
}
\ns

\vers{8}{
Tak tady v cele spí naděje naše posled\\
ní, sedm dní bude prý teď špatné poča\\
sí, jestli se ještě někdy rozední,\\
propusťte Noeho z basy.
}
\vspace{5mm}

\textbf{6. 7. 8. R.}
\newpage
