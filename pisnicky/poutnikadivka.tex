\hyt{poutnikadivka}
\song{Poutník a dívka} \interpret{spiritualkvintet}{Spirituál kvintet}

\vers{1}{
\chord{D}Kráčel krajem poutník, šel sám,\\
\chord{G}kráčel krajem poutník, šel \chord{D}sám,\\
\chord{D}kráčel krajem poutník, kráčel\chord{F\kk} sám\chord{Hm},\\
tu potkal \chord{E}dívku, nesla \chord{E\7}džbán, přistoupil k \chord{A\4}ní \chord{A\7}\mm a pravil:
}

\vers{2}{
\uv{\rep{Ráchel, Ráchel, žízeň mě zmáhá,}\\
tak přistup blíže, nehodná, a dej mi pít.} A ona:
}

\vers{3}{
\uv{\rep{Kdo jsi, kdo jsi, že mi říkáš jménem?}\\
Já tě vidím poprvé, odkud mne znáš?}
}

\vers{4}{
\uv{\rep{Ráchel, Ráchel, znám víc než jméno.}}\\
Pak se napil, ruku zdvih' a kráčel dál.
}

\vers{5}{
\rep{Ten džbán, ten džbán z nepálené hlíny}\\
v onu chvíli zazářil kovem ryzím.
}

\vers{6}{
\rep{Kráčel krajem poutník, šel sám,}\\
ač byl \chord{E}chudý, nepoz\chord{A}nán, přece byl \chord{D}král. \chord{G}\nc\chord{D}
}
\newpage
