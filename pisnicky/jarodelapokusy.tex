\hyt{jarodelapokusy}
\song{Jaro dělá pokusy} \interpret{uhlirsverak}{Uhlíř \& Svěrák}

\vers{1}{
\chord{C}Jaro dělá pokusy, vystrkuje krokusy.\\
Dříve, než se vlády \chord{G}chopí, vystrkuje peri\chord{C}skopy.
}

\vers{2}{
Než se jaro osmělí, vystrkuje podběly.\\
Ty mu asi dolů hlásí čerstvé zprávy o počasí.
}

\refrainn{1}{
\chord{C\7}Jaro! Je to v \chord{F}suchu, zima už \chord{G}nemůže. (zima už \chord{C}nemůže)\\
Stoupá teplota vzduchu a míza do růže. (a míza do růže)\\
Povídám, jaro! Je to v suchu, vichry už nedujou. (vichry už nedujou)\\
Nahlas nebo v duchu lidi se radujou. (lidi se radujou)
}

\vers{3}{
U dopravní cedule vyrostly dvě bledule,\\
blízko telegrafní tyče vyrostly dva petrklíče.
}

\vers{4}{
Mravenci už pracujou, holky sukně zkracujou,\\
slunce svítí, led je tenký, jaro, vem si podkolenky.
}

\refrainn{2}{
Jaro! Je to v suchu, zima už nemůže,\\
stoupá teplota vzduchu a míza do růže.\\
Povídám, jaro! Je to v suchu, vichry už nedujou.\\
Nahlas nebo v duchu lidi se radujou.\\
Stromy se radujou, (stromy se radujou)\\
keře se radujou, (keře se radujou)\\
ryby se radujou, (ryby se radujou)\\
mouchy se radujou, (mouchy se radujou)\\
myši se radujou, (myši se radujou)\\
dveře se radujou, (dveře se radujou)\\
okna se radujou, (okna se radujou)\\
opice se radujou, (opice se radujou)\\
kočky se radujou,\rec{Větší zvířata, ať se radujou} (kočky se radujou)\\
sloni se radujou, \rec{Ještě větší  zvířata.} (sloni se radujou)\\
ministři se radujou. \rec{Nó!} (ministři se radujou)
}
\newpage
