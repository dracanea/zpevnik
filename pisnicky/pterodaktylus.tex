\hyt{pterodaktylus}
\song{Pterodaktylus}

\rec{Když \chord{D}svět kdysi vznik, všude \chord{A}byl jen les, v něm toulalo se zvířat víc, než \chord{D}toulá se dnes.\\
A ten \chord{D}pozemský \chord{D\7}ráj neznal \chord{G}mínus ani \chord{E}plus, a \chord{A}nepotřeboval ani \chord{D}papyrus.}

\refrainn{1}{
\chord{A}Žili tu \chord{D}lvi, lišky, žáby a \chord{A}klokani, netopýři, tapíři i \chord{D}potkani.\\
Jak \chord{D}slepice i \chord{D\7}opice, tak \chord{G}psi i hejno \chord{E}hus, a pak tu také byl \chord{A}pterodakty\chord{D}lus.
}
\vspace{15pt}

\rec{Jednou přiletěl mrak v době polední a z něj někdo křik: \uv{Bude pršet hodně dní!\\
Hej, strejdo Noe, koukej mazat pro pilu, a rychle postav loď velkou jako flotilu!}}

\refrainn{2}{
A vem tam lvy, lišky, žáby a klokany, netopýry, tapíry i potkany.\\
Jak slepice i opice, tak psy a párek hus, jen ať tam nechybí pterodaktylus!
}
\vspace{15pt}

\rec{Tak Noe se zved' a šel domů pro kleště, sám archu postavil, než se dalo do deště.\\
U přístavního můstku pak s lahví rumu stál, jak zvířata šla kolem, tak je počítal.}

\refrainn{3}{
Už je tu pár lišek, lvů, žab a klokanů, netopýrů, tapírů i potkanů.\\
Jak slepice i opice, tak psi a párek hus, ale kdepak, k sakru, je pterodaktylus?
}
\vspace{15pt}

\rec{Když se Noe rozhlíd' deštěm z toho kopce do strání, byli pterodaktylové ještě kdesi schovaní.\\
Noe pustil tedy na loď párek bílých motýlů a zálibně se díval na tu idylu.}

\refrainn{4}{
Tlachání lvů lišek, žáby a klokanů, netopýrů, tapírů i klokanů.\\
Pak zapálil si doutník a dal si Gambrinus, ať si trhne nohou pterodaktylus!
}
\vspace{15pt}

\rec{Vtom kocábka se zvedla, vodní plání vyplula, paní Noemová jedla, ani brvou nehnula,\\
když bílé šňůry deště jako deka přikryly naříkající pterodaktyly.}

\refrainn{5}{
Tak dnes jsou lvi, lišky, žáby a klokani, netopýři, tapíři i potkani.\\
Jak slepice i opice, tak psi a hejno hus, ale ani jediný pterodaktylus.
}
\newpage
