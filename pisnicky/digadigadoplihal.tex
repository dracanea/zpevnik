\hyt{digadigadoplihal}
\song{Diga diga do}
\sub{Karel Plíhal, text Josef Kainar }
\hyl{digadigadoo}{Diga Diga Doo} \hyl{digadigadospiritual}{Diga diga do (Spirituál kvintet)}
\vspace{15pt}

\note{capo 2/3}

\vers{1}{
\chord{Am}Proč bychom se neměli o pohřebním veselí?\\
\chord{F}Diga diga \chord{E}dou, diga \chord{Am}dou dou, \chord{F}diga diga \chord{E}dou, diga \chord{Am}dou.
}

\vers{2}{
Družičky a kaviár, Málaga\footnote{přístavní město na jihu Španělska, rodiště Pabla Picassa} má žár a spár.\\
Diga diga dou, diga dou dou, diga diga dou, diga dou.
}

\refrainn{1}{
\chord{G\bas{G A B H}}Krásnej funus první třídy \chord{C}flétnistky a \chord{C\bas{C C H B}}chlast,\\
\chord{A}pozůstalí nejsou hnidy, \chord{Dm}všude samá \chord{E}vonná mast.
}

\vers{3}{
S tváří mírně napilou nesou dívku spanilou.\\
Diga diga dou, diga dou dou, diga diga dou, diga dou.
}

\vers{4}{
Támhle za tamaryškem apoštolé s Ježíškem.\\
Diga diga dou, diga dou dou, diga diga dou, diga dou.
}

\vers{5}{
Kristus Pán se zadívá na dívku, a pak zazpívá:\\
Diga diga dou, diga dou dou, diga diga dou, diga dou.
}

\refrainn{2}{
Jak zaslechla dívka zpívat jeho božský blues,\\
hejbla kostrou, mrskla pánví, houpla zkrátka na rytmus.
}

\vers{6}{
Z funusu je paseka, celý průvod haleká,\\
diga diga dou, diga dou dou, diga diga dou, diga dou.
} \refsm{2}

\vers{7}{
= \mm\textbf{6.}
}
\newpage
