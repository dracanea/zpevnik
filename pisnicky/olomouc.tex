\hyt{olomouc}
\song{Olomouc} \interpret{plihal}{Karel Plíhal}

\vers{1}{
\chord{G}Občas se mi \chord{Em}stane, což mě \chord{C}vyvede vždy \chord{D}z míry,\\
když mi \chord{Hm\7}někdo řekne, \chord{E\7}pane, ač mám \chord{C}teprv dvacet \chord{A}čtyry.\\
To se \chord{G}asi něco \chord{D}děje, to se \chord{Em}asi něco \chord{C}mění,\\
když si \chord{G}zajdu na k\chord{C}ole\chord{G}je, už tam \chord{C}tolik známých \chord{D}není.
}

\refrain{
Jen tak \chord{C}Olomouc je \chord{D}stále stejná, \chord{Hm\7}holubů se \chord{E\7}snáší hejna\\
\chord{Am}nad radnicí, \chord{D}jež mi vzala \chord{G}kamarády \chord{G\7}ztuhlé v gala.\\
Je \chord{C}stejnej pohled \chord{D}na dvojice, \chord{Hm\7}namačkané \chord{E\7}u Trojice,\\
jen \chord{Am}vzpomínek je \chord{D}mnohem více \chord{C}a příjemně \chord{D}bo\chord{G}lí.
}

\vers{2}{
Občas se mi stane, že se přistihnu, jak kráčím\\
po té cestě vyšlapané, i když stále nevím za čím.\\
To se asi něco děje, to se asi něco mění,\\
více zajímá mě, kde je ta, co zatím se mnou není.
} \refsm{}
\newpage
