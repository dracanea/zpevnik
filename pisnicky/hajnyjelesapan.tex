\hyt{hajnyjelesapan}
\song{Hajný je lesa pán} \interpret{uhlirsverak}{Uhlíř \& Svěrák}
\sub{z pohádky Ať žijí duchové}

\refrainn{1}{
\chord{D}Hajný je \chord{F\kk}lesa \chord{G}pán, \chord{H\7}\nc \chord{Em}zvěří je \chord{E}milo\chord{A\7}ván.\\
\chord{F\kk}Hajný je lesa \chord{Hm}král, a \chord{G}každej pytlák \chord{A}se ho vždycky \chord{D}bál.
}

\vers{1}{
\rec{\chord{B}Co se děje, les mi hoří?\\
Nebo řádí kuny, tchoři?\\
\chord{D}Že by pytlák, lesní pych?\\
Já tu meškám v pantoflích.}\\
\vinv
\chord{B}Klidně si zůstaňte v domácí obuvi,\\
\chord{G}klukům to nevadí, \chord{A}dívky vás omluví.
}

\refrainn{2}{
Máme výborný plán, tímto jste k němu zván,\\
dejte nám stromů pár a dejte nám je třeba jako dar.
}

\vers{2}{
\rec{Sekerečku vem a sekni! Copak to jde? Strom je státní!\\
Zadarmo se nezíská. Z toho koukaj' želízka.}\\
\vinv
Slibuju za kluky, slibuju za holky,\\
za každý strom dáme sto malých do školky.
}

\refrainn{3}{
Tomu já říkám plán, tímto je ujednán.\\
Hajný je lesa pán, je mládeží a zvěří milová\dots
}

\cod{
\dots \chord{F\kk}á - \chord{G}án, \chord{Em} á - \chord{E}á - \chord{A\7}án.\\
\chord{F\kk}Hajný je lesa \chord{Hm}pán, je \chord{G}mládeží a \chord{A}zvěří milo\chord{D}ván.
}
\newpage
