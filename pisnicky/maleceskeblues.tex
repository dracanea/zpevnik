\hyt{maleceskeblues}
\song{Malé české blues}
\sub{aneb doprdeleprace}

\vers{1}{
Koupil jsem \chord{Dm}kozu, nemam kozu, včera mi \chord{Gm}chcipla na virozu,\\
kdybych koupil \chord{B}kozla, taky by mi \chord{A}chcip, takovy jsem \chord{Dm}cyp.\\
\chord{A}Do prdele prace!
}

\vers{2}{
Třešeň v rohu zahrady roky mi nesla, letos uroda na polovinu klesla,\\
že pry špačci, ja ti dam špačci, snědi spoluobčane.\\
Do prdele prace!
}

\vers{3}{
Dřu jak mezek ve zbrojovce, ve volnem čase na zahradě chovam ovce,\\
Neuwirth chova ryby v akvariu, ma na kontě mega a ja nic.\\
Do prdele prace!
}

\vers{4}{
Moje stara, škoda slov, takovy maly Suvorov*,\\
jak mi liskne, tak to třiskne, jak když o traverzu buchne těžky kov.\\
Do prdele prace!
}

\vers{5}{
Starší dcera robi v Londonderry, mladši čeká děcko, ale nevi s kerym,\\
synek sedi, bo byl hlupy, nechal se chytit.\\
Do prdele prace!
}

\vers{6}{
Topolanek, jasne jak facka, jezdi si v plavkach do Toskanska.\\
Ja tak možu na horni splav do Lidečka, tečka.\\
Do prdele prace!
}

\vers{7}{
Na posraneho, aj zachod pada, tady je těžka jaka smysluplna rada,\\
nejlepši bude odložit bendžo a zajit na jedno.\\
Do restaurace, tak ja du.\\
\vinv
\emph{rec:} Vratim se tak v deset.
}

\fakt{Alexandr Vasiljevič Suvorov (1730--1800), slavný ruský vojevůdce, jeden z nejlepších stratégů. V českém jazyce se ustálil jeho citát \textit{\uv{Těžko na cvičišti, lehko na bojišti.}}}
\newpage
