\hyt{kdyznapavoukasedneblues}
\song{Když na pavouka sedne blues}

\vers{1}{
Když \chord{A}na pa\chord{A\7}vouka \chord{D}sedne \chord{D\kk\dm}blues, tak \chord{E}přestane tkát \chord{A}sítě, \chord{E}\\
\chord{A}zale\chord{A\7}ze si \chord{D}do ko\chord{D\kk\dm}uta a \chord{E}pláče jako \chord{A}dítě.\\
\chord{D}Vybrečí se, \chord{A}vysmrká se \chord{H\7}do mušího \chord{E}křídla\\
\chord{A}a pus\chord{A\7}tí se \chord{D}s novou \chord{D\kk\dm}vervou \chord{E}do shánění \chord{A}jídla,\\
\chord{A}a pus\chord{A\7}tí se \chord{D}s novou \chord{D\kk\dm}vervou \chord{E}do shánění \chord{A}jídla.
}

\vers{2}{
Když na velrybu sedne blues, tak ponoří se ke dnu\\
a když to není zrovna v lednu, klidně si k ní sednu.\\
Polechtám ji bublinkama, poplácám po hřbetě\\
a je opět spokojená v tom velrybím světě,\\
a je opět spokojená v tom velrybím světě.
}

\vers{3}{
Když na papouška sedne blues, tak vykašle se na zob\\
a pod zobák si zanadává v rámci slovních zásob.\\
A pak se vrhne na mrkvičku, pokud jsme mu dali\\
a zase je to ten papoušek tak, jak jsme ho znali,\\
a zase je to ten papoušek tak, jak jsme ho znali.
}

\vers{4}{
Když na člověka sedne blues a všechno na něj padá,\\
zakoupí si láhev vína, tklivé písně skládá.\\
Smutek ho však nepřestane pod srdíčkem bodat,\\
dokud se mu nepodaří aspoň jednu prodat,\\
dokud se mu nepodaří aspoň jednu prosadit.
}
\newpage
